\chapter{Workout Generation} \label{workout}
\pagestyle{fancy}

\section{MIA Workout Generation Overview and Introduction}

The workout generation in MIA is created for producing a workout with customization from the user. The generation of a workout within MIA has some dependencies on random value generation and thus is capable of creating different workouts each run. MIA is also capable of creating an entire weeks worth of workouts and outputting it to a file for the user. The entire generation process depends on a few input values, such as maximum number of sets, maximum number of exercises per set, and more which are all defined in a exercises file. Upon running the workout generation, the user enters a difficulty and MIA generates a workout with appropriate difficulty based on this number and the input file (see section \ref{workout algorithm} for more details).  

Throughout this section, workout can be defined as the complete output generated by MIA containing some number of sets, some number of exercises per set, and some number of reps per exercises.

\section{Input File and Defining Workouts}

The MIA workout generation utilizes an input file to determine exercises, exercise weighted values and various generation values. By default, this file is located in ../bin/Resources/InputFiles/exercises.txt but this file name and path can be changed via the MIAConfig file (see section \ref{MIAConfig} for more details). The contents of this input file look similar to the following.

\lstset{language=Python}

\begin{lstlisting}
#============================================================================
# Name        : exercises.txt
# Author      : Antonius Torode
# Date        : created on 3/14/18
# Copyright   : This file can be used under the conditions of Antonius' 
#               General Purpose License (AGPL).
# Description : Different workouts with weighted values for workout generation.
#============================================================================

# Various comments blocks explaining usage.
# ...
# ...
toughness = 0.1
minNumOfExercises = 3.0
maxNumOfExercises = inf
minNumOfSets = 1.0
maxNumOfSets = inf

# Exercises and weights.
push_up = 8.0; reps
sit_up = 15.0; reps
pull_up = 0.75; reps
squat = 3.0; reps
jumping_jack = 30.0; reps
\end{lstlisting}

This file must be in the correct format in order for the MIA workout generation to function properly. First, commented lines are created using the '\#' character. These lines are ignored by MIA when running internal algorithms. Within this input file, spaces are not important. Upon initialization, the MIA program will ignore all spaces within this file. Next, there are a few variables that the user can customize and define within this file which are below.

\begin{lstlisting}
toughness = 0.1
minNumOfExercises = 3.0
maxNumOfExercises = inf
minNumOfSets = 1.0
maxNumOfSets = inf
\end{lstlisting}

These values must appear in the input file before any defined exercises. To begin, toughness is a global variable that helps define the number of reps MIA will output per workout chosen. Increasing this value is a global increase to the workout generation difficulty. The default value for toughness is 0.1 (see section \ref{workout algorithm} for more details). Next, There are minNumOfExercises and maxNumOfExercises variables which are used to determine the minimum number of exercises MIA will choose per set and the maximum number of exercises MIA can choose per set. The maxNumOfExercises value is read in such that a value of 'inf' is allowed. If 'inf' is selected, MIA will set the total number of exercises within the input file to be the maximum. Similarly, there are minNumOfSets and maxNumOfSets values which work in identical ways to minNumOfExercises and maxNumOfExercises only defining a minimum and maximum for number of sets per generated workout instead of number of exercises per set. 

\begin{note}
	At the time of writing this, the MIA program is not designed to account for a value of maxNumOfExercises that is larger than the actual number of exercises defined in the input file.
\end{note}

Following these program variables, the main part of the input file is the defined exercises. The exercises are defined similar to below.

\begin{lstlisting}
# Exercises and weights.
push_up = 8.0; reps
sit_up = 15.0; reps
pull_up = 0.75; reps
squat = 3.0; reps
running = 0.1; mile
jumping_jack = 30.0; reps
\end{lstlisting}

Each exercise is defined using a common form. As shown below, the line must begin with an exercise name. Following this comes an equal sign and then a weighted value. This weighted value is defined to be relative to all other weighted values. This mean that in the above example, the file is claiming 8.0 push ups are equivalent to 15.0 sit ups, and similarly, 0.75 pull ups, etc. The MIA program will assume and each weight value for each exercise is of the same real world difficulty to the user. Following the weighted value must come a semi-colon and then a unit. The equal sign and semi-colon are important because they define how MIA separates the values. As stated previously, spaces are irrelevant in these definitions (See below).

\begin{lstlisting}
# Proper format for definind an exercise in the input file.
exercise_Name = exercise_Weighted_Value; exercise_Unit

#The following three lines are equivalent when read in by MIA.
pull ups = 15.0; reps
pullups=15.0;reps
p   ull u ps   =   15  . 0   ; reps
\end{lstlisting}


\section{Generation Algorithm}\label{workout algorithm}

This section contains an outline of the algorithm used to generate the MIA workouts. The MIA generation is based on creating two curves (maximum and minimum) for a parameter and then deciding upon which parameter to use for a workout by taking a random value between both curves.

\subsection{Number of Sets Per Workout}

The number of sets, denoted $S(s_{min},s_{max},d) \equiv S$ is dependent on three variables. The first two are from the input file which are minNumOfSets, denoted $s_{min}$ and minNumOfSets, denoted $s_{max}$. The last is the difficulty $d$ which is input by the user upon generation. The maximum number of sets was originally based on a linear increase, however for better optimization of the real world workout difficulties, a custom algorithm was created. The maximum $S_{max}(s_{min},s_{max},d)$ and minimum $S_{min}(s_{min},s_{max},d)$ number of sets per workout are given by
\begin{align}
S_{max}(s_{min},s_{max},d) \equiv S_{max}&= \frac{s_{max}-s_{min}}{10^{4/3}} d^{2/3}+s_{min} \\
S_{min}(s_{min},s_{max},d) \equiv S_{min}&= \frac{s_{max}-1.9 \times s_{min}}{1.9 \times 10^{4/3}} d^{2/3}+s_{min}.
\end{align}
Thus since $S(s_{min},s_{max},d)$ is a random value between these curves, we have
\begin{align}
S_{ave}(s_{min},s_{max},d) \equiv S_{ave} &= \frac{S_{max}+ S_{min}}{2} \\ &= \frac{\left(2.9s_{max}-3.8s_{min}\right)}{3.8 \times 10^{4/3}}d^{2/3}+s_{min}.
\end{align}
These are shown in Figure \ref{Svd and Evd}. The algorithm used to determine the sets per workout is identical to that of determining the number of exercises per set.

\begin{figure}[h]
	\centering
	\includegraphics[width=0.5\textwidth]{Images/Svd.png}\includegraphics[width=0.5\textwidth]{Images/Evd.png}
	\caption{Number of sets per workout (left) and number of exercises pet set (right) based on the user input difficulty. The small dotted lines represent the original algorithm which was a simple linear increase in difficulty for each parameter. For the above two figures, values of $s_{min}=1.0$, $s_{max} = 10.0$, $e_{min}=3.0$ and $e_{max}=25.0$ were used.} \label{Svd and Evd}
\end{figure}

\subsection{Number of Exercises Per Set}

The number of exercises, denoted $E(e_{min},e_{max},d) \equiv E$ is dependent on three variables. The first two are from the input file which are minNumOfExercises, denoted $e_{min}$ and minNumOfExercises, denoted $e_{max}$. The last is the difficulty $d$ which is input by the user upon generation. The maximum number of sets was originally based on a linear increase, however for better optimization of the real world workout difficulties, a custom algorithm was created. The maximum $E_{max}(e_{min},e_{max},d)$ and minimum $E_{min}(e_{min},e_{max},d)$ number of sets per workout are given by
\begin{align}
E_{max}(e_{min},e_{max},d) \equiv E_{max}&= \frac{e_{max}-e_{min}}{10^{4/3}} d^{2/3}+e_{min} \\
E_{min}(e_{min},e_{max},d) \equiv E_{min}&= \frac{e_{max}-1.9 \times e_{min}}{1.9 \times 10^{4/3}} d^{2/3}+e_{min}.
\end{align}
Thus since $E(e_{min},e_{max},d)$ is a random value between these curves, we have
\begin{align}
E_{ave}(e_{min},e_{max},d) \equiv E_{ave} &= \frac{E_{max}+ E_{min}}{2} \\ &= \frac{\left(2.9e_{max}-3.8e_{min}\right)}{3.8 \times 10^{4/3}}d^{2/3}+e_{min}.
\end{align}
These are shown in Figure \ref{Svd and Evd}.


\subsection{Number of Reps Per Exercise}

The number of reps, denoted $R(t,d) \equiv R$ is dependent on two variables. The first is toughness $t$ which is gathered from the input file or defaults to 0.1. The second is difficulty $d$ which is input by the user upon generation.



\section{Real World Difficulties} 

