\chapter{Commands and Syntax}
\pagestyle{fancy}

\section{Valid Syntax}
MIA is designed to be used similar to a terminal or command prompt. One enters commands and the uses the 'Enter' key to perform the commands. MIA commands are NOT case sensitive. By default, all commands are changed to lower case before executing through the MIA program. If a command is not typed exactly how it is intended (including spaces and newline characters) it may not execute.


\section{Complete List of Valid Commands (CLVC)} \label{CLVC}

\begin{lstlisting}
help
\end{lstlisting}
Displays a valid lists of commands and a brief description to go along with each.
\begin{lstlisting} 
add   
\end{lstlisting}
Adds two positive integers of any length. This adds two strings together using a similar algorithm one would when adding large numbers by hand. It is possible to get results by entering non-number entries but will serve no significance due to the way MIA internally converts strings to integers by shifting the ASCII values.
\begin{lstlisting} 
collatz   
\end{lstlisting}
Produces a collatz sequence based on a specified starting integer. This method uses the logn data type which means if a number of the sequence extends the storage of a long, the results will become untrustworthy. 
\begin{lstlisting} 
crypt -d0s1   
\end{lstlisting}
Encrypts a string using the d0s1 algorithm. This is explained more in chapter \ref{D3C}.
\begin{lstlisting} 
crypt -d0s2  
\end{lstlisting}
Encrypts a string using the d0s2 algorithm. This is explained more in chapter \ref{D3C}.
\begin{lstlisting} 
decrypt -d0s1   
\end{lstlisting}
De-crypts a string using the d0s1 algorithm. This is explained more in chapter \ref{D3C}.
\begin{lstlisting} 
decrypt -d0s2   
\end{lstlisting}
De-crypts a string using the d0s2 algorithm. This is explained more in chapter \ref{D3C}.
\begin{lstlisting} 
digitsum 
\end{lstlisting}
Returns the sum of the digits within an integer of any size. Similar to the add command, this converts a string to an array of integers using ASCII shifting and then sums the values together. Due to this, you can also find values for entering non-numerical strings.
\begin{lstlisting} 
factors   
\end{lstlisting}
Returns the number of factors within an integer. The integer must be smaller than C++'s internal storage for the long data type.
\begin{lstlisting} 
lattice   
\end{lstlisting}
Returns total lattice paths to the bottom right corner of an n x m grid. This function is only valid for situations in which the answer will not exceed the internal storage of a long data type.
\begin{lstlisting} 
multiply  
\end{lstlisting}
Multiplies two integers of any length. Similar to add, this multiplies two strings together using a similar algorithm one would when multiplying large numbers by hand. It is possible to get results by entering non-number entries but will serve no significance due to the way MIA internally converts strings to integers by shifting the ASCII values.
\begin{lstlisting} 
palindrome   
\end{lstlisting}
Determines if a positive integer is palindrome. The integer must be smaller than C++'s internal storage for the long data type.
\begin{lstlisting} 
prime   
\end{lstlisting}
Determines if a positive integer is prime or not. The integer must be smaller than C++'s internal storage for the long data type.
\begin{lstlisting} 
prime -f   
\end{lstlisting}
Determines all of the prime factors of a positive integer. The integer must be smaller than C++'s internal storage for the long data type.
\begin{lstlisting} 
prime -n  
\end{lstlisting}
Calculates the n'th prime number up to a maximum number of 2147483647.
\begin{lstlisting} 
prime -n -p   
\end{lstlisting}
Creates a file of all prime numbers up to a maximum number of 2147483647.
\begin{lstlisting} 
prime -n -c   
\end{lstlisting}
Clears the file created by 'prime -n -p'.
\begin{lstlisting} 
subtract   
\end{lstlisting}
Finds the difference between two integers of any length. Similar to add, this subtracts two strings together using a similar algorithm one would when subtracting large numbers by hand. It is possible to get results by entering non-number entries but will serve no significance due to the way MIA internally converts strings to integers by shifting the ASCII values.
\begin{lstlisting} 
triangle   
\end{lstlisting}
Determines if a number is a triangle number or not. The integer must be smaller than C++'s internal storage for the long data type.
\begin{lstlisting} 
exit  
\end{lstlisting}
Quits MIA. 